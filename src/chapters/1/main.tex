\chapter{GIỚI THIỆU ĐỀ TÀI}

Trong bối cảnh nền y tế ngày càng phát triển và hiện đại hóa, việc quản lí hiệu quả thông tin bệnh nhân đóng vai trò then chốt trong nâng cao chất lượng chăm sóc sức khỏe. Mỗi bệnh nhân đều có một hồ sơ bệnh án riêng, chứa đựng toàn bộ lịch sử khám chữa bệnh, chẩn đoán, điều trị và theo dõi sức khỏe. Tuy nhiên, với khối lượng thông tin lớn và phức tạp, phương pháp quản lí truyền thống trên giấy tờ dễ dẫn đến sai sót, mất mát dữ liệu và tốn nhiều thời gian tra cứu.

Chính vì vậy, nhu cầu xây dựng một hệ thống quản lí bệnh án điện tử hiện đại, tiện lợi và chính xác là vô cùng cần thiết. Thế nên, phầm mềm "Quản lí bệnh án bệnh nhân" - \textbf{MeReMa} được phát triển nhằm đáp ứng những yêu cầu đó.

\section{Mục đích}

Phần mềm "Quản lí bệnh án bệnh nhân" - MeReMa là một giải pháp tin học hóa hỗ trợ các cơ sở y tế trong việc quản lí thông tin bệnh nhân và bệnh án của họ một cách hiệu quả và chính xác. Ngoài việc quản lí bệnh án, phần mềm còn hỗ trợ một số tính năng khác như quản lí lịch hẹn khám, thống kê và báo cáo,... nhằm tối ưu hóa quy trình làm việc của nhân viên y tế và nâng cao trải nghiệm của bệnh nhân.

Về phần tài liệu này, mục đích là cung cấp một cái nhìn tổng quan, dễ hiểu về các yêu cầu, thành phần của dự án. Tài liệu sẽ được cung cấp cho các bên liên quan như giảng viên hướng dẫn, sinh viên thực hiện và các bên liên quan khác trong quá trình phát triển phần mềm. 

\section{Phạm vi}

Đề tài tập trung phát triển một phần mềm hỗ trợ quản lý bệnh án điện tử trong phạm vi nội bộ của cơ sở y tế như trung tâm y tế hoặc bệnh viện có quy mô từ nhỏ đến trung bình.

Hệ thống không đi sâu vào các chức năng quản lý toàn bộ hoạt động của bệnh viện (ví dụ như tài chính, kho thuốc, lịch trực...), mà chỉ tập trung vào khía cạnh quản lý bệnh án và hồ sơ bệnh nhân. Ngoài ra, hệ thống cũng không bao gồm việc tích hợp với hệ thống y tế quốc gia hay các chuẩn dữ liệu y tế quốc tế, tuy nhiên vẫn có thể mở rộng để hỗ trợ trong các phiên bản phát triển tiếp theo.

\section{Từ điển thuật ngữ}

\section{Tài liệu tham khảo}

\begin{thebibliography}{9}
    \bibitem{IEEE830}
    IEEE Recommended Practice for Software Requirements Specifications," in IEEE Std 830-1998, vol., no., pp.1-40, 20 Oct. 1998.

\end{thebibliography}

\section{Thông tin chung về phần mềm}

\subsection{Môi trường phát triển phần mềm dự kiến}

\begin{itemize}
    \item Hệ điều hành: Linux (Arch Linux và NixOS)
    \item Trình soạn thảo và IDE: Visual Studio Code, Neovim và Android Studio
    \item Công cụ hỗ trợ: Git 
\end{itemize}

\subsection{Công nghệ sử dụng dự kiến}

\begin{itemize}
    \item Ngôn ngữ lập trình chính: Dart và Golang
    \item Framework: Flutter và Gin
    \item Hệ quản trị cơ sở dữ liệu: PostgreSQL
\end{itemize}

\section{Tổng quan}

Tài liệu được viết dựa vào chuẩn IEEE 830-1998 \cite{IEEE830} với nhiều sự điều chỉnh để phù hợp hơn với yêu cầu của đồ án môn học \textbf{Nhập môn công nghệ phần mềm - SE104}. Nội dung tài liệu được chia thành các chương như sau:
\begin{itemize}
    \item Chương 1: Giới thiệu đề tài - Giới thiệu tổng quan về đề tài, mục đích, phạm vi và các thông tin chung về phần mềm.
    \item Chương 2: Các nhóm tính năng chính của hệ thống - Mô tả chi tiết các nhóm tính năng và tính năng chính mà hệ thống phải cung cấp cho người dùng. Nội dung mô tả bao gồm: Giới thiệu tính năng, Quy trình và Các yêu cầu chức năng tương ứng.
\end{itemize}