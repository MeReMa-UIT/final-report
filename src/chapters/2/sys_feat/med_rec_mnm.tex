\section{Quản lí hồ sơ bệnh án}

Nhóm chức năng ``Quản lí hồ sơ bệnh án'' là nhóm chức năng chính của hệ thống, với vai trò đặc biệt quan trọng. Nhóm chức năng gồm các chức năng sau:
\begin{itemize}
    \item Thêm hồ sơ bệnh án
    \item Cập nhật hồ sơ bệnh án
    \item Tra cứu hồ sơ bệnh án
    \item Xóa hồ sơ bệnh án
\end{itemize}

\subsection{Thêm hồ sơ bệnh án}

\noindent \textbf{Giới thiệu tính năng}
Tính năng ``Thêm hồ sơ bệnh án'' cho phép người dùng thêm hồ sơ bệnh án từ bệnh án có sẵn hoặc tạo bệnh án mới.

\noindent \textbf{Quy trình}
\begin{itemize}
    \item Người dùng đăng nhập vào hệ thống.
    \item Người dùng truy cập vào giao diện ``Thêm hồ sơ bệnh án''.
    \item Người dùng chọn thêm từ bệnh án có sẵn hoặc nhập bệnh án mới.
    \item Người dùng nhập các thông tin bệnh án.
    \begin{itemize}
        \item Nếu thêm thành công: Lưu trữ hồ sơ bệnh án vào hệ thống.
        \item Nếu thêm không thành công: Yêu cầu sửa / nhập lại.
    \end{itemize}
\end{itemize}

\noindent \textbf{Các yêu cầu chức năng}
\begin{itemize}
    \item \textbf{REQ\_01} Hồ sơ bệnh án phải gắn với bệnh nhân hợp lệ trong hệ thống.
\end{itemize}

\subsection{Cập nhật hồ sơ bệnh án}

\noindent \textbf{Giới thiệu tính năng}
Tính năng ``Cập nhật hồ sơ bệnh án'' cho phép người dùng cập nhật thông tin bệnh, thay đổi chẩn đoán,...

\noindent \textbf{Quy trình}
\begin{itemize}
    \item Người dùng đăng nhập vào hệ thống.
    \item Người dùng truy cập vào giao diện ``Tra cứu hồ sơ bênh án''
    \item Người dùng tìm bệnh án cần cập nhật.
    \item Người dùng lựa chọn ``Cập nhật hồ sơ bệnh án'' và cập nhật các thông tin cần thiết.
    \begin{itemize}
        \item Nếu thành công: Lưu trữ bệnh án đã cập nhật vào hệ thống.
        \item Nếu không thành công: Yêu cầu sửa / nhập lại.
    \end{itemize}
\end{itemize}

\noindent \textbf{Các yêu cầu chức năng}
\begin{itemize}
    \item \textbf{REQ\_01} Các thông tin được cập nhật phải hợp lệ.
    \item \textbf{REQ\_02} Hệ thống phải lưu lại đầy đủ lịch sử thay đổi của bệnh án. (Sử dụng trường ``version'' để lưu trữ lịch sử phiên bản.)
\end{itemize}

\subsection{Tra cứu hồ sơ bệnh án}

\noindent \textbf{Giới thiệu tính năng}
Tính năng ``Tra cứu hồ sơ bệnh án'' cho phép người dùng thực hiện việc tìm kiếm hồ sơ bệnh án nhanh chóng và chính xác.

\noindent \textbf{Quy trình}
\begin{itemize}
    \item Người dùng đăng nhập vào hệ thống.
    \item Người dùng truy cập vào giao diện ``Tra cứu hồ sơ bệnh án''.
    \item Người dùng nhập các tiêu chí tìm kiếm như ``Tên bệnh nhân'',...
    \begin{itemize}
        \item Nếu tìm thấy: Hiển thị danh sách bệnh án đã tìm thấy.
        \item Nếu không tìm thấy: Hiển thị ``Không tìm thấy''.
    \end{itemize}
\end{itemize}

\noindent \textbf{Các yêu cầu chức năng}
\begin{itemize}
    \item \textbf{REQ\_01} Hệ thống phải thực hiện việc tìm kiếm nhanh chóng và chính xác.
    \item \textbf{REQ\_02} Hệ thống phải tuân thủ phân quyền khi thực hiện tìm kiếm.
\end{itemize}

\subsection{Xóa hồ sơ bệnh án}

\noindent \textbf{Giới thiệu tính năng}
Tính năng ``Xóa hồ sơ bệnh án'' cho phép người dùng thực hiện thao tác xóa các bệnh án khi vô tình trùng lặp hoặc bệnh án giả lập,...

\noindent \textbf{Quy trình}
\begin{itemize}
    \item Người dùng đăng nhập vào hệ thống.
    \item Người dùng truy cập vào giao diện ``Tra cứu hồ sơ bệnh án''.
    \item Người dùng tìm bệnh án cần xóa.
    \item Người dùng lựa chọn xóa đơn thuốc.
    \item Hệ thống hiển thị cảnh báo xác nhận trước khi xóa.
    \begin{itemize}
        \item Nếu thành công: Hệ thống cập nhật trạng thái bệnh án thành ``Đã xóa''.
        \item Nếu không thành công: Hệ thống hiển thị ``Xóa không thành công''.
    \end{itemize}
\end{itemize}

\noindent \textbf{Các yêu cầu chức năng}
\begin{itemize}
    \item \textbf{REQ\_01} Hệ thống phải yêu cầu xác nhận lại trước khi xóa.
    \item \textbf{RER\_02} Hệ thống phải lưu trữ bệnh án đã xóa với trạng thái ``Đã xóa'' và không được thực xóa bệnh án khỏi hệ thống.
\end{itemize}